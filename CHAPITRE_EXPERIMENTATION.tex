\chapter{Expérimentation}

Ce chapitre présente l'expérimentation de l'application CollectEVP avec un jeu de données réalistes. L'objectif est de démontrer le fonctionnement complet du système de validation hiérarchique des éléments variables de paie, depuis la saisie par le gestionnaire jusqu'à la validation finale par les responsables et la consultation par les ressources humaines.

\section{Contexte de l'expérimentation}

\subsection{Environnement de test}

L'expérimentation a été réalisée dans un environnement de développement configuré avec les technologies suivantes :

\begin{itemize}
    \item \textbf{Frontend} : Application React développée avec Vite, accessible sur \texttt{http://localhost:5173}
    \item \textbf{Backend} : API Symfony 6.4, accessible sur \texttt{http://127.0.0.1:8080/api}
    \item \textbf{Base de données} : SQLite avec un jeu de données de test incluant 50 employés répartis sur 3 divisions (Production, Qualité, Logistique)
    \item \textbf{Utilisateurs de test} : 5 utilisateurs représentant chaque niveau hiérarchique du système
\end{itemize}

\subsection{Scénario d'utilisation}

Le scénario d'expérimentation simule un cycle complet de gestion des EVP :

\begin{enumerate}
    \item Un \textbf{Gestionnaire} saisit des primes et congés pour plusieurs employés de son équipe
    \item Les soumissions sont validées ou rejetées par le \textbf{Responsable Service}
    \item Les EVP approuvés par le service sont ensuite validés ou rejetés par le \textbf{Responsable Division}
    \item Le \textbf{RH} consulte le reporting global et gère la base de données des employés
    \item L'\textbf{Administrateur} gère les utilisateurs du système
\end{enumerate}

\section{Guide d'utilisation par rôle}

\subsection{Interface Gestionnaire}

Le gestionnaire est responsable de la saisie et de la soumission des éléments variables de paie pour les employés de son équipe.

\subsubsection{Connexion}

\begin{figure}[H]
    \centering
    \includegraphics[width=0.8\textwidth]{screenshots/01_login.png}
    \caption{Page de connexion - Authentification par email et mot de passe}
    \label{fig:login}
\end{figure}

La figure~\ref{fig:login} présente la page de connexion de l'application. L'utilisateur saisit son email et son mot de passe. Le système authentifie l'utilisateur via l'API backend et génère un token JWT valide pendant une heure.

\textbf{Compte de test} : \texttt{gestionnaire@ocp.ma} / \texttt{password123}

\subsubsection{Saisie des primes}

\begin{figure}[H]
    \centering
    \includegraphics[width=0.9\textwidth]{screenshots/02_gestionnaire_saisie_prime.png}
    \caption{Interface de saisie des primes - Formulaire de calcul automatique}
    \label{fig:saisie_prime}
\end{figure}

La figure~\ref{fig:saisie_prime} illustre l'interface de saisie des primes. Le gestionnaire peut :

\begin{itemize}
    \item Sélectionner un employé dans la liste déroulante
    \item Saisir les critères de calcul (taux monétaire, nombre de postes, scores, etc.)
    \item Visualiser le montant calculé automatiquement par l'application
    \item Ajouter la prime à la liste des soumissions en attente
\end{itemize}

Le système calcule automatiquement le montant de la prime selon la formule :
\[
Montant = TauxMonétaire \times NombrePostes \times ScoreÉquipe \times NoteHiérarchique \times ScoreCollectif
\]

\subsubsection{Saisie des congés}

\begin{figure}[H]
    \centering
    \includegraphics[width=0.9\textwidth]{screenshots/03_gestionnaire_saisie_conge.png}
    \caption{Interface de saisie des congés - Calcul automatique de l'indemnité}
    \label{fig:saisie_conge}
\end{figure}

La figure~\ref{fig:saisie_conge} présente le formulaire de saisie des congés. Le gestionnaire peut :

\begin{itemize}
    \item Définir les dates de début et de fin du congé
    \item Le nombre de jours est calculé automatiquement
    \item Saisir la tranche et l'indemnité forfaitaire
    \item Optionnellement, demander une avance sur congé
    \item L'indemnité totale est calculée automatiquement
\end{itemize}

\subsubsection{Soumission et historique}

\begin{figure}[H]
    \centering
    \includegraphics[width=0.9\textwidth]{screenshots/04_gestionnaire_validation.png}
    \caption{Tableau des soumissions en attente - Actions de soumission globale}
    \label{fig:validation_gestionnaire}
\end{figure}

La figure~\ref{fig:validation_gestionnaire} montre le tableau des soumissions préparées. Le gestionnaire peut :

\begin{itemize}
    \item Soumettre individuellement chaque EVP
    \item Utiliser le bouton "Soumettre tout pour validation" pour traiter toutes les soumissions en une seule action
    \item Consulter l'historique de toutes ses soumissions
    \item Voir les commentaires de rejet éventuels dans l'onglet historique
\end{itemize}

\begin{figure}[H]
    \centering
    \includegraphics[width=0.9\textwidth]{screenshots/05_gestionnaire_historique.png}
    \caption{Onglet historique - Suivi des soumissions avec statuts et commentaires}
    \label{fig:historique_gestionnaire}
\end{figure}

La figure~\ref{fig:historique_gestionnaire} présente l'onglet historique où le gestionnaire peut :

\begin{itemize}
    \item Filtrer par type (Prime, Congé, Tous)
    \item Filtrer par statut (Soumis, Modifié, À revoir)
    \item Consulter les commentaires de rejet du responsable service
    \item Modifier et resoumettre les demandes rejetées
\end{itemize}

\subsection{Interface Responsable Service}

Le responsable service valide ou rejette les soumissions des gestionnaires de son service.

\subsubsection{Tableau de validation}

\begin{figure}[H]
    \centering
    \includegraphics[width=0.9\textwidth]{screenshots/06_respo_service_validation.png}
    \caption{Interface de validation service - Liste des soumissions en attente}
    \label{fig:validation_service}
\end{figure}

La figure~\ref{fig:validation_service} présente l'interface de validation du responsable service. Pour chaque soumission, le responsable peut :

\begin{itemize}
    \item Consulter les détails complets de la prime ou du congé
    \item Voir les commentaires de rejet éventuels de la division (si la demande a été rejetée par la division et renvoyée)
    \item Valider ou rejeter chaque type (Prime et Congé) indépendamment
    \item Ajouter un commentaire en cas de rejet
\end{itemize}

\subsubsection{Processus de validation}

\begin{figure}[H]
    \centering
    \includegraphics[width=0.7\textwidth]{screenshots/07_respo_service_dialog_validation.png}
    \caption{Dialog de validation - Confirmation avec option de commentaire}
    \label{fig:dialog_validation_service}
\end{figure}

La figure~\ref{fig:dialog_validation_service} montre le dialog de validation. Le responsable service peut :

\begin{itemize}
    \item Valider la demande, qui passe alors au niveau division
    \item Rejeter avec un commentaire explicatif, renvoyant la demande au gestionnaire
    \item Traiter Prime et Congé séparément si la soumission contient les deux types
\end{itemize}

\subsubsection{Historique des validations}

\begin{figure}[H]
    \centering
    \includegraphics[width=0.9\textwidth]{screenshots/08_respo_service_historique.png}
    \caption{Historique des validations - Suivi de toutes les soumissions traitées}
    \label{fig:historique_service}
\end{figure}

La figure~\ref{fig:historique_service} présente l'historique complet des soumissions avec filtres par type et par statut.

\subsection{Interface Responsable Division}

Le responsable division valide ou rejette les soumissions approuvées par le responsable service.

\subsubsection{Validation niveau division}

\begin{figure}[H]
    \centering
    \includegraphics[width=0.9\textwidth]{screenshots/09_respo_division_validation.png}
    \caption{Interface de validation division - Validation finale des EVP}
    \label{fig:validation_division}
\end{figure}

La figure~\ref{fig:validation_division} montre l'interface de validation du responsable division. Les soumissions affichées ont déjà été validées par le service. Le responsable division peut :

\begin{itemize}
    \item Valider définitivement la demande
    \item Rejeter avec un commentaire, renvoyant la demande au responsable service
    \item Consulter l'historique complet des validations
\end{itemize}

\subsubsection{Historique division}

\begin{figure}[H]
    \centering
    \includegraphics[width=0.9\textwidth]{screenshots/10_respo_division_historique.png}
    \caption{Historique division - Vue complète des validations avec filtres}
    \label{fig:historique_division}
\end{figure}

La figure~\ref{fig:historique_division} présente l'historique avec des filtres avancés permettant de suivre l'évolution des soumissions.

\subsection{Interface RH}

Les ressources humaines ont accès à une vue globale du système et gèrent la base de données des employés.

\subsubsection{Reporting global}

\begin{figure}[H]
    \centering
    \includegraphics[width=0.9\textwidth]{screenshots/11_rh_reporting.png}
    \caption{Reporting global RH - Vue consolidée de tous les EVP validés}
    \label{fig:reporting_rh}
\end{figure}

La figure~\ref{fig:reporting_rh} présente le reporting global accessible au RH. Cette interface permet de :

\begin{itemize}
    \item Consulter tous les EVP validés de toutes les divisions
    \item Filtrer par division, service, type (Prime/Congé), statut
    \item Rechercher par matricule, nom, prénom
    \item Exporter les données (fonctionnalité prévue)
\end{itemize}

\subsubsection{Gestion des employés}

\begin{figure}[H]
    \centering
    \includegraphics[width=0.9\textwidth]{screenshots/12_rh_gestion_employes.png}
    \caption{Gestion des employés - CRUD complet de la base de données}
    \label{fig:gestion_employes}
\end{figure}

La figure~\ref{fig:gestion_employes} illustre l'interface de gestion des employés. Le RH peut :

\begin{itemize}
    \item Ajouter de nouveaux employés
    \item Modifier les informations existantes
    \item Supprimer des employés
    \item Filtrer et rechercher dans la liste
\end{itemize}

\subsubsection{Traitement des demandes}

\begin{figure}[H]
    \centering
    \includegraphics[width=0.9\textwidth]{screenshots/13_rh_demandes_employes.png}
    \caption{Traitement des demandes d'ajout d'employés - Demandes des gestionnaires}
    \label{fig:demandes_employes}
\end{figure}

La figure~\ref{fig:demandes_employes} montre l'interface de traitement des demandes d'ajout d'employés formulées par les gestionnaires. Le RH peut approuver ou rejeter chaque demande.

\subsection{Interface Administrateur}

L'administrateur gère les utilisateurs du système et la configuration.

\subsubsection{Gestion des utilisateurs}

\begin{figure}[H]
    \centering
    \includegraphics[width=0.9\textwidth]{screenshots/14_admin_utilisateurs.png}
    \caption{Gestion des utilisateurs - CRUD complet avec activation/désactivation}
    \label{fig:admin_utilisateurs}
\end{figure}

La figure~\ref{fig:admin_utilisateurs} présente l'interface de gestion des utilisateurs. L'administrateur peut :

\begin{itemize}
    \item Créer de nouveaux utilisateurs avec attribution de rôle et division
    \item Modifier les informations des utilisateurs existants
    \item Activer ou désactiver des comptes
    \item Supprimer des utilisateurs
\end{itemize}

\subsubsection{Ajout d'un utilisateur}

\begin{figure}[H]
    \centering
    \includegraphics[width=0.7\textwidth]{screenshots/15_admin_ajout_utilisateur.png}
    \caption{Dialog d'ajout d'utilisateur - Formulaire de création}
    \label{fig:ajout_utilisateur}
\end{figure}

La figure~\ref{fig:ajout_utilisateur} montre le formulaire de création d'utilisateur avec tous les champs requis : nom, email, rôle, division et mot de passe.

\subsubsection{Permissions par rôle}

\begin{figure}[H]
    \centering
    \includegraphics[width=0.9\textwidth]{screenshots/16_admin_permissions.png}
    \caption{Tableau des permissions - Vue d'ensemble des droits par rôle}
    \label{fig:permissions}
\end{figure}

La figure~\ref{fig:permissions} présente le tableau récapitulatif des permissions associées à chaque rôle dans le système.

\section{Scénario complet d'utilisation}

Cette section présente un scénario complet illustrant le flux de validation hiérarchique.

\subsection{Étape 1 : Saisie par le gestionnaire}

Le gestionnaire \textit{Ahmed Bennani} (Production) saisit une prime pour l'employé \textit{Khalid Mansouri} :

\begin{itemize}
    \item Taux monétaire : 500 DH
    \item Nombre de postes : 2
    \item Score équipe : 85\%
    \item Note hiérarchique : 4/5
    \item Score collectif : 90\%
    \item Montant calculé : 3,060 DH
\end{itemize}

La prime est soumise et apparaît dans l'historique avec le statut "Soumis".

\subsection{Étape 2 : Validation par le responsable service}

Le responsable service \textit{Fatima Zahra Alami} consulte la soumission et valide la prime. Le statut passe à "Validé Service" et la demande est transmise au responsable division.

\subsection{Étape 3 : Validation par le responsable division}

Le responsable division \textit{Hassan Mouhib} examine la demande et la valide définitivement. Le statut final devient "Validé Division".

\subsection{Étape 4 : Consultation par le RH}

Le RH \textit{Mohammed Tazi} consulte le reporting global et peut voir la prime validée dans le tableau consolidé de toutes les divisions.

\section{Résultats de l'expérimentation}

\subsection{Couverture fonctionnelle}

L'expérimentation a permis de valider les fonctionnalités suivantes :

\begin{itemize}
    \item ✅ Authentification sécurisée par JWT
    \item ✅ Saisie et calcul automatique des primes
    \item ✅ Saisie et calcul automatique des congés
    \item ✅ Soumission individuelle et globale
    \item ✅ Validation hiérarchique à 5 niveaux
    \item ✅ Rejet avec commentaires et retour au niveau inférieur
    \item ✅ Gestion indépendante de Prime et Congé dans une même soumission
    \item ✅ Filtrage et recherche avancée
    \item ✅ Gestion complète des employés (CRUD)
    \item ✅ Gestion complète des utilisateurs (CRUD)
    \item ✅ Reporting consolidé multi-divisions
    \item ✅ Historique complet avec traçabilité
\end{itemize}

\subsection{Performance et ergonomie}

\begin{itemize}
    \item \textbf{Temps de réponse} : Les opérations CRUD s'exécutent en moins de 200ms
    \item \textbf{Interface utilisateur} : Navigation intuitive avec feedback visuel immédiat
    \item \textbf{Calculs automatiques} : Les montants sont calculés en temps réel lors de la saisie
    \item \textbf{Gestion des erreurs} : Messages d'erreur clairs et explicites
\end{itemize}

\subsection{Points forts identifiés}

\begin{enumerate}
    \item \textbf{Séparation des types} : La gestion indépendante de Prime et Congé permet une flexibilité maximale dans le workflow de validation
    \item \textbf{Traçabilité} : L'historique complet permet de suivre l'évolution de chaque soumission
    \item \textbf{Interface moderne} : L'utilisation de Tailwind CSS et shadcn/ui offre une expérience utilisateur professionnelle
    \item \textbf{Sécurité} : L'authentification JWT et le contrôle d'accès par rôle garantissent la sécurité des données
\end{enumerate}

\section{Conclusion}

L'expérimentation démontre que l'application CollectEVP répond aux besoins fonctionnels définis. Le système de validation hiérarchique fonctionne correctement, permettant une gestion efficace des éléments variables de paie avec une traçabilité complète. L'interface utilisateur est intuitive et les performances sont satisfaisantes pour un usage en environnement de production.

Les captures d'écran présentées dans ce chapitre illustrent le fonctionnement complet de l'application et peuvent servir de guide d'utilisation pour les futurs utilisateurs du système.

